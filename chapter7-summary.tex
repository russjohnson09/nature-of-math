\documentclass[12pt]{article}
%\usepackage{geometry}
\usepackage[a4paper,centering,includehead]{geometry}
%\usepackage[paperwidth=6in, paperheight=9in]{geometry} %For Book Printing
\usepackage{changepage}
\usepackage{color}
\usepackage[usenames,dvipsnames,svgnames,table]{xcolor}
%==============================
\usepackage{microtype}
%==============================
%==============================
\usepackage{babel}
\usepackage{csquotes}
\MakeOuterQuote{"}
%==============================
%Math Related Packages
\usepackage{amsmath, amsfonts, amssymb, amsthm, xfrac, mathtools}
\begin{document}
\begin{flushright}
Russ Johnson\\
Homework 1\\
\today\\
\end{flushright}

In the early 1600s, two mathematicians from the British Isles, John Napier (15550-1617) and Henry Briggs (1561-1631), invented the "logarithm." The concept of the logarithm would have a tremendous impact on the mathematical community both from a practical and theoretical standpoint. Logarithms can be utilized to simplify otherwise tedious computations in multiplication, division and the extraction of roots. For example the calculation of $\sqrt[7]{234.65}$ would be incredibly tedious without the help of logarithms. Pierre-Simon Laplace observed that this new invention of Napier and Briggs' doubled the productivity of an astronomer.

As the century progressed, three French mathematicians commanded
the spotlight. One was Ren\'{e} Descartes ( 1 596- 1 650) , a philosopher and
mathematician whose 1637 {\it Discours de la m\'{e}thode} became a landmark
in the history of philosophy. 

\end{document}