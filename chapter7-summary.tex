\documentclass[12pt]{article}
%\usepackage{geometry}
\usepackage[a4paper,centering,includehead]{geometry}
%\usepackage[paperwidth=6in, paperheight=9in]{geometry} %For Book Printing
\usepackage{changepage}
\usepackage{color}
\usepackage[usenames,dvipsnames,svgnames,table]{xcolor}
%==============================
\usepackage{microtype}
\usepackage{setspace}
%==============================
%==============================
\usepackage[english]{babel}
\usepackage{csquotes}
\MakeOuterQuote{"}
%==============================
%Math Related Packages
\usepackage{amsmath, amsfonts, amssymb, amsthm, xfrac, mathtools}
\begin{document}
\begin{flushright}
Russ Johnson\\
Chapter 7 Summary\\
\today\\
\end{flushright}

\begin{doublespace}
In the early 1600s, two mathematicians from the British Isles, John Napier (1550-1617) and Henry Briggs (1561-1631) invented the "logarithm." The concept of the logarithm would have a tremendous impact on the mathematical community both from a practical and theoretical standpoint. Logarithms can be utilized to simplify otherwise tedious computations in multiplication, division and the extraction of roots. For example the calculation of $\sqrt[7]{234.65}$ would be incredibly tedious without the help of logarithms. Pierre-Simon Laplace observed that this new invention of Napier and Briggs' doubled the productivity of an astronomer.

In the later half of this century, three French mathematicians made major contributions. One of these three was Ren\'{e} Descartes (1596- 1650). Descartes was a philosopher and mathematician whose 1637 {\it Discours de la m\'{e}thode} became an important historical text in philosophy. This text was hotly debated amgonst philosophers, but it is one of the appendices that had the biggest impact on mathematics. In {\it La g\'{e}om\'{e}trie} Descartes described what we now call analytic geometry. He described the union of algebra and geometry as its own field of mathematics. According to Descartes, analytic geometry would be indispensable in all subsequent mathematical work.

Our next mathematician is Blaise Pascal (1623-1662) who had a very short mathematical career. He was a child prodigy and invented a calculating machine. He looked for omens in everyday objects and had an interest in theology. He decided that mathematics was not what God intended for him to do and only briefly returned to the field of mathematics when he was 35.

We also have the very interesting Pierre de Fermat (1601-1665) of Toulouse. Fermat made many significant discoveries in a wide range of mathematics. He created his own version of analytic geometry independent of Descartes, and in some ways Fermat's version is closer to what we use today. Descartes, however, was the first to publish and so he reaps most of the glory for its invention. Fermat is infamous for claiming to have proofs for different results, but leaving the work of writing these proofs down to later mathematicians. The most well known example of this is Fermat's last theorem. This was found scribbled in Fermat's personal copy of {\it Arithmetica}.
\end{doublespace}
\vspace{-5mm}
\begin{quote}
But it is impossible to divide a cube into two cubes, or a fourth power into two fourth powers, or generally any power beyond the square into two like powers; of this I have found a remarkable demonstration. This margin is too narrow to contain it.
\end{quote}
\vspace{-5mm}
\begin{doublespace}
Euler, who provided proofs for many of Fermat's "theorems," was not able to proof this assertion for anything greater than $n=4$. Fermat's last theorem has only recently been solved using methods that were not available while Fermat was alive.

Finally, we have the main mathematician for this chapter, Isaac Newton. Newton was born on December $25^{th}$. He was in a very frail state, but managed to make it through the harsh Lincolnshire winter, and would live to be 84. He lived with his grandmother after his mother remarried. He kept to himself most of the time as a kid as well as when he was a student at Cambridge. Despite being well-known today, Newton remained almost completely anonymous during his time at Cambridge. Even during the time that is now referred to as his wonderful years in which he had a great burst of creativity, he was unknown.

One of Newton's many contributions was the discovery of a general formula for $(x+y)^n$. This is the Binomial Theorem
\[(x+y)^n = \sum_{k=0}^n {n \choose k}x^{n-k}y^k. \]
Pascal noted that the coefficients could be easily obtained from the array now known as  "Pascal's triangle":
\end{doublespace}
\begin{center}
\small\addtolength{\tabcolsep}{-5pt}
\begin{tabular}{cccccccccccccccccccccc}
   &    &    &    &    &    &    &    &    &    &  1\\
   &    &    &    &    &    &    &    &    &  1 &    &  1\\
   &    &    &    &    &    &    &    &  1 &    &  2 &    &  1\\
   &    &    &    &    &    &    &  1 &    &  3 &    &  3 &    &  1\\
   &    &    &    &    &    &  1 &    &  4 &    &  6 &    &  4 &    &  1\\
   &    &    &    &    &  1 &    &  5 &    & 10 &    & 10 &    &  5 &    &  1\\
   &    &    &    &  1 &    &  6 &    & 15 &    & 20 &    & 15 &    &  6 &    &  1\\
   &    &    &  1 &    &  7 &    & 21 &    & 35 &    & 35 &    & 21 &    &  7 &    &  1\\
   &    &  {\bf 1} &    & {\bf 8} &    & {\bf 28} &    & {\bf 56} &    & {\bf 70} &    & {\bf 56} &    & {\bf 28} &    & {\bf 8} &    & {\bf 1}\\
   &  1 &    &  9 &    & 36 &    & 84 &    & 126 &    & 126 &    & 84 &    & 36 &    &  9 &    &  1\\
 1 &    & 10 &    & 45 &    & 120 &    & 210 &    & 252 &    & 210 &    & 120 &    & 45 &    & 10 &    &  1\\
\end{tabular}\\
and so on.
\end{center}
Pascal's triangle can be used for the following
\[(a+b)^8 = a^8 + 8a^7b + 28a^6b^2 + 56a^5b^3 + 70a^4b^4 + 56a^3b^5 + 28a^2b^6 + 8ab^7 + b^8.\]
You will notice that the coefficients are equal to the entries in the boldfaced row of Pascal's triangle.

\end{document}
