\documentclass[10pt]{article}
\usepackage[a4paper,centering,includehead]{geometry}
%==============================
%\usepackage{microtype}
\usepackage{setspace}
\usepackage{natbib}
%==============================
%Math Related Packages
\usepackage{amsmath, amsfonts, amssymb, amsthm, xfrac, mathtools}
\begin{document}
\begin{flushright}
Russ Johnson\\
The Mathematization of Science\\
\today\\
\end{flushright}

\begin{doublespace}
During the seventeenth century Descartes and Galileo revolutionized the scientific method. Descartes believed that the properties of physics could be explained using only the fundamental properties of shape, extension, and motion in space and time. Also, because shape can be described using only extension, Descartes asserted, "Give me extension and motion and I shall construct the universe." Descartes believed in a mechanical world which meant that the laws of mechanics could be used to described all things. Furthermore, because the laws of mechanics could be described using mathematics all things must also be mathematically describable. It was this philosophy that influenced many scientists of this century and led to a huge advancement in the scientific method. One of the many scientists that Descartes left an impression on was Newton who went on to discover the laws of motion.

Galileo advocated the Copernican theory and other controversial scientific theories of the time. He would not back down or stop publishing even after receiving threats from the Spanish Inquisition. Stemming from his pioneer spirit, Galileo was also an incredible inventor. Although he did not invent the telescope, his insight allowed him to construct one simply from hearing the idea.  Galileo invented the pendulum clock and a new type of compass that did certain numerical computations. Galileo also studied sound and suggested wave theory as an explanation to the origin of sound. 

Many of Galileo's believes on nature mirror those of Descartes. He believed that nature followed simple and consistent patterns. God created the universe using rigorous mathematics and it was only after much labor that mathematicians are able to derive these laws of the universe. Following from this belief, Galileo was convinced that physics had a set of axioms that controlled how it acted. God did not intervene with physics and this laws would remain constant. Unlike the axioms of geometry, however, these axioms were not always self-evident without experimentation. Many times Galileo was able to arrive at these axioms using only mind experiments and only used experiments in the real world to convince people of there truth. 

One of the major criticisms of Galileo by his peers was that he did not try to explain why these laws are the way they are. Galileo did not try to explain what weight was or why it would effect the movement of an object in the way it does. Even Descartes protested and said, "Everything that Galileo says about bodies falling in empty space is built without foundation: he ought first to have determined the nature of weight." However, Galileo was right to not start with this as the foundation as he would probably never have been able to get past this seemingly simple property. Instead, Galileo created a very high level of abstraction for the movement of objects. By assuming that they moved in a vacuum, he was able to create mathematical formulas to describe them. After these mathematical formulas are available for us to use, we are able to introduce other things that might effect motion, like air resistance, and trace the paths of bodies in motion so accurately that we can create a trajectory of a bullet travelling over a mile. And so, we produce incredibly powerful methods for describing even complex natural phenomenon by breaking down physics to its most elemental form.



\end{doublespace}

\end{document}


Maybe I should include some more stuff on Newton?
