\documentclass[12 pt]{article}
%\usepackage{geometry}
\usepackage[a4paper,centering,includehead]{geometry}
%\usepackage[paperwidth=6in, paperheight=9in]{geometry} %For Book Printing
%==============================
%\setlength{\parskip}{14pt plus 1pt minus 2pt}
\usepackage{microtype}
%==============================
%Math Related Packages
\usepackage{amsmath, amsfonts, amssymb, amsthm, xfrac, mathtools}
%==============================
\usepackage{framed}
%==============================
\usepackage{pdflscape}
%==============================
% For more complex numbering
%\usepackage{enumitem}
\usepackage{enumerate}
\usepackage{mdwlist}
%==============================
%For Importing inkscape svg files with latex font.
\usepackage{import}
%==============================
%Table Related
\usepackage{multirow, multicol, tabularx, longtable}
\usepackage[tableposition=below]{caption}
\captionsetup[longtable]{skip=1em}
%==============================
%Colors and Graphics
\usepackage{color}
\usepackage{graphicx}
%Hyperref
\usepackage[unicode=true, pdfusetitle,
 bookmarks=true,bookmarksnumbered=false,bookmarksopen=false,
 breaklinks=false,pdfborder={0 0 0},backref=false,colorlinks=false]
 {hyperref}
\usepackage{cite}
\usepackage{tocloft}
\usepackage{tikz}
%==============================
%==============================
\begin{document}
\begin{flushright}
Russ Johnson \\
Homework 4\\
\today \\
\end{flushright}
%
\subsection*{Euclid's Third Book}

<p>
Book II explored what we now call "geometric algebra." That is, it
framed in geometric terms certain relationships that now are most easily
translated into algebraic equations . Of course, the notion of algebra was
foreign to the Greeks, and its appearance as a formal system lay centuries
in the future.
</p>

<p>
PROPOSITION 11.4 If a straight line be cut at random, the square on the
whole is equal to the squares on the segments and twice the rectangle
contained by the segments .
</p>

<p>
This is written more simply as $(a + b)^2= a^2 +b^2 + 2 ab = a^2 + 2ab + b^2$
</p>

<p>
Yet the equivalence of his geo­metric statement and its algebraic counterpart is clear. Much of Book I I
was o f this nature .
</p>

<p>
nor did Book I I I contain the familiar results for a cirde's circumference
$( C = \pi D)$ or area $(A = \pi r^2)$ . A full treatment of these latter topics would have to wait
until Archamedias.
</p>

<p>
In fact, while Euclid did not explicitly say so, most subsequent mathe­
maticians assumed that his were the only constructible regular polygons
and that any others were simply beyond the capability of compass and
straightedge.
</p>

<p>
It was thus a shock of monumental proportions when the teenaged
Carl Friedrich Gauss discovered how to construct a regular heptadeca­
gon ( 17-gon) in 1796.
</p>

<a href="http://www.jimloy.com/">
  <img title="Powered by MathJax"
    src="http://www.jimloy.com/geometry/17-gon.gif"
    border="0" alt="17-gon" />
</a>

<p>
Similar rectilineal figures are such as have their angles
severally equal and the sides about the equal angles proportional .
</p>

<p>
Making use of the Eudox­
ean theory of Book V, Euclid proved, in Proposition VI .4, that if two tri­
angles have their corresponding angles equal, then their corresponding
sides must be proportional ; conversely, in Proposition V1 . 5 , he showed
that if two triangles have their sides proportional, then their correspond­
ing angles must be equal .
</p>

<p>
If in a right-angled triangle a perpendicular be drawn
from the right angle to the base, the triangles adjoining the perpendic­
ular are similar both to the whole and to one another.
</p>

<p>
A critical definition was that of a prime number, that is,
a number greater than 1 that is divisible by (Euclid said, "measured by")
only 1 and itself.
</p>


\end{document}
